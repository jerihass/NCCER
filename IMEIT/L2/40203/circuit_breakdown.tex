\documentclass[12pt]{article}
\usepackage{adjustbox}
\usepackage{circuitikz}
\pagestyle{empty}
\textwidth 6in
\oddsidemargin 0.25in
\topmargin -0.25in
\textheight 8.5in
\begin{document}
\begin{flushleft}

Steps for reducing resistances:
\begin{minipage}[t]{\linewidth}
	\adjustbox{valign=t}{%
	%problem 3
\begin{circuitikz}[american voltages]
\draw
	(0,0) to [short, -] (0, 0)
	to [R, l=R$_1$] (2.0, 0)
	to [short] (2.0, 0)
	to [short] (2.0, 0.5)
	to [short] (2.0, 0.5)
	to [R, l=R$_2$] (4.0, 0.5)
	to [short] (4.0, 0.5)
	to [short] (4.0, 0.0)
	to [short] (4.0, 0.0)
	to [R, l=R$_3$] (6.0, 0)
	to [short, -] (6.0, 0)
	(2.0, 0) to [short] (2.0, -0.5)
	to [short] (2.0, -0.5)
	to [R, l_=R$_4$] (4.0, -0.5)
	to [short] (4.0, -0.5)
	to [short] (4.0, -0.0)
	(0,0) to [short] (0, -2)
	to [R, l=R$_5$] (3.0, -2)
	to [short] (3.0, -2)
	to [short] (3.0, -1.5)
	to [short] (3.0, -1.5)
	to [R, l=R$_6$] (5.0, -1.5)
	to [short] (5.0, -2.0)
	to [short] (6.0, -2.0)
	(3.0,-1.5) to [short] (3.0, -2.5)
	to [short] (3.0, -2.5)
	to [R, l_=R$_7$] (5.0, -2.5)
	to [short] (5.0, -2.5)
	to [short] (5.0, -1.5)
	(6.0, -2.0) to [short] (6, 0)
	(-.5, -1) to [short, *-] (0, -1)
	(6, -1) to [short, -*] (6.5, -1);
\end{circuitikz}
}
\end{minipage}
\begin{enumerate}
\medskip
\item Find smallest groups. In this case, R$_2$ with R$_4$ and R$_6$ with R$_7$ are parallel groups so must be reduced first.
\begin{minipage}[t]{\linewidth}
	\adjustbox{valign=t}{%
	%problem 3
\begin{circuitikz}[american voltages]
\draw
	(0,0) to [short, -] (0, 0)
	to [R, l=R$_1$] (2.0, 0)
	to [short] (2.0, 0)
	to [R, l=R$_{2,4}$] (4.0, 0.0)
	to [short] (4.0, 0.0)
	to [short] (4.0, 0.0)
	to [R, l=R$_3$] (6.0, 0)
	to [short, -] (6.0, 0)
	(0,0) to [short] (0, -2)
	to [R, l=R$_5$] (3.0, -2)
	to [short] (3.0, -2)
	to [short] (3.0, -2)
	to [short] (3.0, -2)
	to [R, l=R$_{6,7}$] (6.0, -2)
	(6.0, -2.0) to [short] (6, 0)
	(-.5, -1) to [short, *-] (0, -1)
	(6, -1) to [short, -*] (6.5, -1);
\end{circuitikz}

}
\end{minipage}
\medskip
\item In the next step R$_1$, R$_{2,4}$ and R$_3$ are in series, as are R$_5$ and R$_{6,7}$.
\begin{minipage}[t]{\linewidth}
	\adjustbox{valign=t}{%
	\begin{circuitikz}[american voltages]
\draw
	(0,0) to [short, -] (0, 0)
	to [R, l=R$_{1,2,4,3}$] (6.0, 0.0)
	(0,0) to [short] (0, -2)
	to [R, l=R$_{5,6,7}$] (6.0, -2)
	(6.0, -2.0) to [short] (6, 0)
	(-.5, -1) to [short, *-] (0, -1)
	(6, -1) to [short, -*] (6.5, -1);
\end{circuitikz}
}
\end{minipage}
\medskip
\item In the next step R$_{1,2,4,3}$ and R$_{5,6,7}$ are parallel, so reduce those.

\begin{minipage}[t]{\linewidth}
	\adjustbox{valign=t}{%
	\begin{circuitikz}[american voltages]
\draw
	(-.5,0) to [short, *-] (0, 0)
	to [R, l=R$_{1,2,3,4,5,6,7}$] (6.0, 0.0)
	to [short, -*] (6.5, 0);
\end{circuitikz}
}

\item The final reduced resistance is represented by a single resistance.

\end{minipage}
\end{enumerate}
\end{flushleft}
\end{document}

